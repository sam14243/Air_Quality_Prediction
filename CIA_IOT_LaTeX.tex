\documentclass{article}
\usepackage[utf8]{inputenc}
\usepackage{multirow}
\title{Using IoT and Cloud Computing for Smart Agriculture}
\author{}
\date{18th January, 2023}

\begin{document}

\maketitle

\section{Topic Summary} 
\begin{flushleft}
Agriculture is one of the major sources for any of the largest population countries like India, China etc. to earn money and carry out the livelihood. So, the main problem addressed here is “How to improve the crop production by controlling the cost, monitoring performance and maintenance by the involvement of IoT and Cloud Computing?”. Autonomous techniques such as Robotics and satellite technology were already used but were not efficient in small scale. So, the author introduces the concept of Drones which are Unmanned Aerial Vehicles (UAVs) with the combination of IoT and Cloud computing technologies, which can help in real-time data extraction, evaluation and solutions to the agricultural farming.\\
IOT is a concept which describes a system where everyday objects, such as home appliances, furniture, clothes, etc are readable, recognizable, locatable, addressable and/or controllable via the Internet through combining technological developments in item identification, sensors and wireless sensor networks, embedded systems and nanotechnology. Cloud Computing describes a computing paradigm which provide on-demand access, either the application data or the storage space to large pool of system which are connected to each other. Most of the papers in the literature review of Smart Agricultures reports the results of IoT systems which are focused on sensing and monitoring, while actuation and remote control is much less addressed. The author discusses a new technique of monitoring the crop field with the help of Smart Drones by remotely controlling it and monitoring the required parameters. \\
The proposed model is involvement of IoT and Cloud Computing in Drones, giving them a notion of “Smart”. The technology used is Sky Drone FPV2 and compresses of a camera module, a data module and a 4G/LTE modem. It has an embedded software for flight planning and control based on GPS navigation, as well as google maps and is furnished with the complete ICT (Information and Communication Technologies) equipment for data processing from sensor devices. It consists of 3 modules – Sensing, Communication and Coordination. \\
\begin{itemize}
\item Synchronous flight and navigation sensors like vision sensor, gyroscope, tilt and current sensors are installed which aid the navigation and monitor the environment of the drone in order to detect and avoid unexpected obstacles. The camera sensor should be at least 12-megapixel resolution with different frequency which are used to identifying pests, weeds and diseases, estimating crop yield, etc. Visual cameras are used during day time and thermal cameras are used during night.
\item The Smart Drone will have an antenna with nearly isotropic radiation intensity patterns and has 4G/LTE modem with embedded IoT wireless technologies like Wi-Fi and ZigBee, a computation, storage and actuation capabilities. The communication devices connect the drone to the IoT devices which can be mobiles, laptops, etc. which acts as a gateway and provide internet access. The data collected can be sent to cloud at any time.
\item Coordination consists of 3 components; Mission Control which takes user input and dispatches it to desired components, Mission Planning which breaks down the high-level tasks to flight routes and Sensor Data Analysis which mosaics the images taken by different sensors and combines them into one image.
\end{itemize} \par
The phases of the process flow are pre-planning the path and defining the search engine for the drone, searching the path while images are captured, detection of any uncertainties in the field, streaming the captured images on the IoT devices at real time, analysis and evaluation of the data in the cloud system, and cultivation control based on the information received which help farmers predict their fields and the next steps to be performed. \\
The advantage of using cloud computing storage server is that the cost of data as a service is reduced. The information relevant can be stored in a single centralized location and they can be accessed anytime from any location which also helps in addressing problems that occur in specific processes of production. \\
In conclusion, the Smart Drone is used to fly on an agricultural land to detect the required parameters which help in building a sustainable smart agricultural practice by preventing the outcomes of any disasters, prediction of field and best practices to be followed in the farm land. Features such including a pluggable scheduler or an intelligent analyser along with Security and risk features are to be added. \\
\end{flushleft}
\section{Key Contributions}
\begin{itemize}
    \item Adoption of IoT and Cloud Computing in Agriculture to make smarter decisions, reduce cost, provide efficiency and boost production.
    \item Techniques such as robotics and satellite technology were already used and this paper introduces to the use of drones.
    \item Drones also work better in small scale and gets an overall survey of the area.
    \item The combination of IoT and cloud computing with drones, provides insights of the crops like plant health indices, plant counting, plant height field prediction, mapping of field water, drainage, etc.
    \item Internet of Things (IoT) describes a system which connects the world’s objects in both sensory and intelligent manner through combining technological developments in item identification, sensors and wireless sensor networks, embedded systems and nanotechnology.
    \item Cloud Computing describes a large pool of connected systems which provide a new way of adding, using and exchanging IT service based on the internet which involves providing dynamic, expandable and virtualized resources.
    \item The literature reviewed focuses on sensing and monitoring, while actuation and remote control are also addressed in this model.
    \item The aim of the model is to provide an intelligent cultivation control for the farmers using drones with the involvement of IoT and Cloud Computing.
    \item The drone contains an embedded software for flight planning and control based on GPS navigation, as well as Google maps and is furnished with the complete ICT equipment for data processing.
    \item The proposed model consists of 3 modules – Sensing, Communication and Coordination.
    \item Flight sensor, navigation sensors like vision sensor, gyroscope, tilt, current sensors and camera sensors are installed in the drone.
    \item The drone has 4G/LTE modem with embedded IoT wireless technologies like Wi-Fi, ZigBee which help in wireless communication and information is stored in the cloud in real-time.
    \item The data collected is stored in the application cloud and used for data analysis, evaluation and provide best practices and techniques for a certain situation to the farmers.
    \item The coordination block takes the user input dispatches it to components, the high level tasks are broken down to flight routes and the various images are merged to one.
    \item The phases are as follows: Pre-planning, searching, detection, streaming, data analysis, and cultivation control.
    \item Several features such as an intelligent analyser in the drone, pluggable scheduler and various security and risk features can be added.
\end{itemize}
\section{My views}
Agriculture is one the important sources of livelihood and income for many of the developing countries with high population. In this paper, the author have discussed a model which consists of a Smart Drone which integrates IoT and Cloud Computing in helping to build a sustainable Smart Agriculture. The architecture and the process flow of the model is well described. After thoroughly reading and understanding this paper, this model of Smart Drones can definitively be an advancement in the field of Smart Agriculture. However, this model has not been implemented in this paper. The cost and the detailed specifications of the drone components is not discussed yet. In conclusion, this model may change the notion of Agriculture, if it is made efficient in the future.
\section{Agreements}
\begin{center}
\begin{tabular}{|c|} 
 \hline
 \\
 \multirow{4}{28em}{“Also, adoption of Internet of Things (IoT) and Cloud Computing in any area are leading them to a notion of “Smart” like Smart Health Care systems, Smart Cities, Smart Mobility, Smart Grid, Smart Home and Smart Metering etc."} \\
 \\
 \\
 \\
 \\
 \hline
 \\
 \multirow{3}{28em}{I agree with this statement as the domain of AI, IoT and Cloud Computing is developing every day. The application of IoT and in every day tasks will make the human life much easier.} \\
 \\ 
 \\
 \\
 \hline
\end{tabular}
\end{center}

\begin{center}
\begin{tabular}{|c|} 
 \hline
 \\
 \multirow{4}{28em}{“To face these challenges with limited natural resources available, we need to involve the
emerging technologies like Internet of Things (IoT) and Cloud Computing into agricultural sector along with various machinery or devices.”} \\
\\
\\
\\
\\
\hline
\\
\multirow{6}{28em}{I agree with this statement because the environment of agriculture is highly dynamic. The Conditions required for each crops and the cases of natural calamities highly differ. By involving modern technologies, they are providing new approaches to farmers to make smarter decisions and reducing cost.} \\
\\
\\
\\
\\
\\
\\
\hline
\end{tabular}
\end{center}
\section{Pitfall(s)}
\begin{center}
\begin{tabular}{|c|} 
 \hline
 \\
 \multirow{6}{28em}{“The goal of smart agriculture with the advent of IoT is to provide latest technology in agriculture and farming for better crop production by gathering the existing real-time status of crop and make the farmers understand the advancement in agriculture, with lot of added features and benefits in order to improve the farming practices.”} \\
 \\
 \\
 \\
 \\
 \\
 \\
 \hline
 \\
 \multirow{5}{28em}{I partially agree with this statement because the model does help in a better sustainable smart agriculture. However, usually farmers are uneducated and they live in isolated villages. They have to be trained to use and understand the purpose of this model. Also, the cost of the drone which is integrated with IoT and Cloud may not be feasible to them.} \\
 \\ 
 \\
 \\
 \\
 \\
 \\
 \hline
\end{tabular}
\end{center}

\section{Submission by:}
Name: Anantha Murthy S \\
\\
Class: AI \& DS - 'A' \\
\\
Reg No: 21011101019
\end{document}
